In this section the project's guidelines for handling documents is documented. 

\subsection{Placing of documents} 
All documents are to be placed in General $\rightarrow$ files $\rightarrow$  \(<\)applicable folder\(>\). In this way we assure that all project members have an easy time locating and accessing important documents. All project teams have their own folder where they may set up a tree structure that they feel is well suited for their needs as long as the tree structure is easy to navigate through. There are also different folders where material that is not group specific is placed. Documents to placed in these folder could for example be meeting presentations, agendas, surveys etc. 

\subsection{Living documents} 
The living documents are to be produced with a start in the beginning of iteration 1. A copy of the current version of the document is to be put in the General files in Teams at the start of iteration 2, 3 and 4 as well as after the project is finished. At the start of each Iteration when the new version of a living document is finished, the corresponding reader and inspector is contacted by the person responsible for writing the document. The reader and inspectors of the different living documents are: 
\begin{itemize}
    \item Architecture notebook: Filip Eriksson
    \item Software requirement specification: Gustav Karlsson
    \item Education plan: Sam Anlér
    \item Project plan: Max Karlsson and Patrik Palmgren
    \item Quality assurance plan: Axel Trolme 
\end{itemize}
The people responsible for making sure that the inspection of the documents is done are: 
\begin{itemize}
    \item Patrik Palmgren
    \item Emma Johansson
\end{itemize}
Inspection of the living documents is done in order to ensure that they are up to date and in line with what we want the documents to fulfill. During the iterations it is important to write down all improvements of the processes so that they are continually documented.

\subsection{Version handling of documents} 
The document is preferably to be edited in Overleaf or in another editor that handles LaTeX code. The LaTeX code is thereafter pushed to a child branch of master where all the commits are to be pushed to. This ensures that the group is able to refer back to certain commits that cohere to a specific version of the document. 