In this section, processes for how the management team is working will be further described. 

\subsection{Updates and weekly meetings}
Every Monday, lunch meetings are to be held from 12.15 - 13.00 where projects matters are discussed. These meetings are held in a SCRUM-fashion, where all the attendances are to describe what they've done the last week and what they are to do the upcoming week. In this way, no one is left behind and a continuous flow of tasks can be exercised. If there are any major deadlines during the week or any other major happenings, these questions are to be brought to the table as well. Please see section 5.2 for a detailed plan of how communication with the other teams are exercised. 

\subsection{Communication with other teams}
In order to have a good work flow throughout the entire company, it's very important that the communication between the different teams works efficiently. Having the communication to fail entails the happening of sub-optimization within the different teams with a strictly negative impact for the project as a whole. Below are a couple of communication-related processes described, exercised within the management team. 

\begin{itemize}
    \item Max Klasson is responsible for communicating with our lead analyst Sara Lindholm when needed.
    \item Axel Trolme communicates with the management team on a daily basis, as well as with the team leaders needed for the moment being. The main purpose for this communication is to get the information needed in order to be bale to update our documents accordingly and to make sure that the project are on time as planned. 
    \item Patrik Palmgren revises all our living documents and communicates with the persons responsible for writing the living documents, e.g Axel Trolme, Emma Johansson, Sara Lindholm, Martin Friberg, to make sure that they meet their intended quality and has the correct information in them.
    \item Martin Friberg is responsible for communicating with the development team when needed. 
    \item Max Klasson is responsible for communicating with the validation team when needed. 
    \item Sam Anlér is responsible for the creation and summary of status reports. See section 5.3 below.

\end{itemize}

\subsection{Status reports}
Every other week, status reports are sent out to every single member of the organization as mentioned above. The purpose of these status reports are to get a better picture of the health within the organization, to make sure that the responsibilities are assigned properly and helps the management team to allocate resources to the correct tasks currently exercised. Example of questions being answered in the status reports are as follows:

\begin{itemize}
    \item How is your current workload? Too low, too high, or good?
    \item Do you have a good understanding for the task(s) currently occupied with? If no, what would you like to have help with?
    \item Do you have any major concerns regarding the project for the moment?
    \item Is there anything else you would like to bring to the attention for the management team?
\end{itemize}

These surveys are then followed up by communicating directly with each individual in the organization regarding their answers, discussing any kind of problems and or challenges. The result from these discussions and the surveys are later summarized and action is taken by the management team accordingly. The main purpose of pursuing these surveys are to make sure that we'll have a healthy organization where everyone can feel that their workload are manageable. 

\subsection{Revision of actions plans}
The management team is responsible for making sure that the project is finished on time with an accepted result. In order to fulfill the responsibility, the management team revises all our gantt-charts on a weekly basis, making sure that we're on time with the development of the product and with the tests etcetera. If the project for some reason can't keep up to the pace needed to follow the gantt-charts and overall plans for the project, actions are taken accordingly. New plans are then developed and communicated to the different teams. 

\subsection{How to obtain project goals}
In order to obtain project goals on time, the management group keeps a continuous discussion with every team within the organization to make sure that they have the situation under control. Having this continuous communication, any troubles or problems can be avoided or managed in time before any major issues occurs. 