In this section, processes regarding the analysis team are described.

\subsection{Requirement Procurement}

The procurement of requirements is started with a thorough study of the documentation uploaded to the company by the customer, Region Östergötland. Through this a first set of requirements is created. These requirements are then assessed and evaluated during meetings with Region Östergötland and the Product Manager. 

\subsection{Requirement formalisation}
The draft requirements agreed upon in meetings mentioned above are then formalised into the SRS accordingly to the course specifications.

\subsection{Requirement Evaluation and Improvement}
The formalised requirements are to iteratively be discussed with the customer as well as internally to asses if changes in the customer needs or the project scope have made requirements obsolete or unattainable. An example of a process for this is through user tests with nurses. The requirements are then to be updated in living documents.

\subsection{Documentation}
The documentation of the requirements is based on a living SRS-document as well as a RTM where the requirements are described and mapped to the system and tests,

\subsection{Categorisation of Requirements}
The requirements are to be categorised based on where in the product the fit in and also what priority level for the project it has. E.G. Req 1 - Patient view, High priority.

\subsection{Communication of Requirements to the Testing team}
Meetings are to be held with both teams to make sure that the Testing team have a grasp on the requirements they are supposed to test. When the process of Analysis is over the Analysis team is also to become a part of the testing team which will naturally enable clear communication and good understanding.

\subsection{Validation of requirement fulfillment}
This process is left to the Testing team.
