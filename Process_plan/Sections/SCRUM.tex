\subsection{Sprints}
The project is divided into specific time frames called iterations. More on this topic and what is to be done during the different sprints can be found in the project plan. 

\subsection{Sprint planning}
Project manager Axel Trolme has together with the team leaders created a GANTT-chart of the different tasks that is to be performed in the specific sprints. The mentioned GANTT-charts can be found in the project plan. The sprint planning makes the entire group get an overview what is to be done in different phases of the project. 

\subsection{Product Backlog}
Since the software requirements specification(SRS) is currently under update, so is also the product backlog for the project. The product backlog is a list of tasks used by the development team in order to achieve the goal product and is located in the "Development" folder in Teams. The tasks in the product backlog shall be broken down into small parts that together makes up the web application. In the near future, the product backlog shall to cover how to implement all the requirements in the SRS and the items in the product backlog is also to map against a specific requirement from the SRS. The product backlog shall be listed as issues in GitLab where each developer can choose an issue that is reasonable to implement at that point in time. Issues for each sprint has been produced from the GANTT-chart in the project plan which in its turn has been produced from the SRS together with the Hi-Fi prototype. The Hi-Fi prototype is used as a reference for what different features in the application should look like as well as for how a feature should work. The functionality in the Hi-Fi prototype was also double checked through the SRS so that all the functional requirements were implemented in the prototype.  When adding an issue to GitLab, it shall also contain a mapping to the number of the requirements that it is supposed to cover. Each issue should be handpicked in consideration of what else is currently under implementation. The issues are also to be divided into groups based on importance. The developers are to mainly choose issues that are of a higher importance to the project.  

\subsection{Standups}
To get a better flow in the processes standups are to be used. The different groups are to do a standup in the beginning of every meeting, where each member of the team is to explain what they have done since the last meeting as well as what they are to do until the next meeting. The standups are supposed to ensure transparency in the groups as well as enhance the group's productivity. During the meetings, problems that have emerged during the passed week are discussed and help is given from the other members of the team. The development team is to have standups at least twice a week, on mondays and wednesdays. They are also to have an oppurtunity for feedback on thursdays during the obligatory meeting times. All of the teams are also working with continuous communication through the Microsoft Teams chat. In short, the main topics discussed are on the standups are: 
\begin{itemize}
    \item What did I work on yesterday?
    \item What am I working on today?
    \item What issues are blocking me?
\end{itemize}



\subsection{SCRUM-master}
The team leaders for the different teams are to work more as SCRUM-masters since we want a way of leading people without having formal authority over them. The SCRUM-master's responsibilities is to follow our SCRUM guidelines as strictly as possible. The SCRUM master invites to the development meetings as well as sets the agenda for the meeting where he makes sure that everybody in the group is heard and gets to share their work and problems. 


