\setlength{\parindent}{0pt}
The text below describes the background to the project HeartByte is currently working on, to develop a self-care system for Region Östergötland.

\subsection{Customer introduction \& main focus}
HeartByte has got a new project with the customer Region Östergötland, a region within the Swedish healthcare system who sees a possibility in extending the self-care treatment for a chosen group of patients with chronic diseases. With this system, Region Östergötland believe they can reduce the occupancy rate at the hospitals as an effect of patients not having to visit the hospital recurrently. \vspace{5mm}

Region Östergötland says these patients can be monitored by letting them selves do their samplings of tests at home, and given care through a web based system that HeartByte is to deliver. In order to create customer value, Region Östergötland needs a system to handle and monitor many patients at once with focus on good usability for the employees and care givers.

\subsection{Relevant constraints}
The product delivered to Region Östergötland is be delivered for free. Due to this, HeartByte does not have an ordinary budget to keep track off with cash equivalents. Instead, HeartByte has a budget made up of hours that has to be taken in consideration. \vspace{5mm}

HeartByte has put together a group of 26 employees with complementary skills with 160 hours each to spend throughout the entire project. This leaves the budget at $26 \cdot 160 = 4160$ hours. Furthermore, there is a list of general constraints for the final product to meet:

\begin{itemize}
    \item The system has to be available for fixed Windows workstations.
    \item The system has to be available for tablets. 
    \item It has to be a customizeable dashboard available, meaning that the care giver to a limited extent should be able to decide what information and modules that is to be presented for the user.  
    \item When accessing data from the patient records, Open EHR APIs has to be used
    \item When accessing data from the patient records, the action has to be logged in the system for an admin user to retrieve if needed.
\end{itemize}

A detailed list of functional and non-functional requirements applicable to the product is to be found in the project's SRS-document.

\subsection{Project goal}
The goal of the project is to deliver a product satisfying the requirements given by the customer. In order to assure the final product meeting the customers quality demands, sub-goals for the project has been set. See the table below for more information.

\begin{table}[]
\begin{tabular}{|l|l|}
\hline
\multicolumn{1}{|c|}{\textbf{Goal}}& \multicolumn{1}{c|}{\textbf{Meaning}}\\ \hline
Implement continuous delivery & Make possible to external stakeholders to monitor the progress \\ \hline
\begin{tabular}[c]{@{}l@{}}Use a CI-environment for the used \\ repository\end{tabular} &\begin{tabular}[c]{@{}l@{}}Mitigate and avoid major 
issues when pushing new code to\\ the repository by running unit tests for intended functionality\end{tabular}\\ \hline
Document all processes & \begin{tabular}[c]{@{}l@{}}To create a reliable \& effective structure of the project,\\ its processes should be documented and updated\end{tabular}\\ \hline
Deliver weekly status reports & \begin{tabular}[c]{@{}l@{}}To keep the entire organization and external stakeholders\\ updated regarding current tasks and progression\end{tabular}\\ \hline
Use a bug-tracking system& \begin{tabular}[c]{@{}l@{}}To make the process of debugging easier. Git issues is \\ to be used\end{tabular}\\ \hline
\begin{tabular}[c]{@{}l@{}}Use a traceability system for \\ tasks and requirements\end{tabular} & \begin{tabular}[c]{@{}l@{}}To increase the satisfaction of the customer, a traceability-\\ system between features, tasks and requirements is to be \\ used. GitLab is to fulfill this task.\end{tabular} \\ \hline
\begin{tabular}[c]{@{}l@{}}Internal followup on relevant \\ documents\end{tabular} & \begin{tabular}[c]{@{}l@{}}In order to increase the value of having living documents, \\ they also have to be updated accordingly as the project \\ makes progress.\end{tabular} \\ \hline
\end{tabular}
\caption{\label{tab:table-name}Internal project goals.}
\end{table}
\subsection{Start and expected end date}
\begin{itemize}
    \item The start of this project is 2020-09-09
    \item Intended and planned code stop set to 2020-12-04
    \item The delivery of final end-product at VSSE 2020-12-10
\end{itemize}