
\documentclass{article}

\usepackage[utf8]{inputenc}
\usepackage{graphicx}
\usepackage{comment}
\usepackage{hyperref}
\usepackage{tabularx}
\usepackage{booktabs}
\usepackage{makecell}
\usepackage{multirow}
\usepackage[title]{appendix}
\graphicspath{./Pictures/}

\newcolumntype{n}{>{\hsize=0.7\hsize \raggedright\arraybackslash}X}
\newcolumntype{w}{>{\hsize=1.6\hsize \raggedright\arraybackslash}X}
\newcolumntype{C}{>{\centering \hsize=1\hsize}X}

\begin{document}
	% Title Page
	\maketitle
\setlength{\parskip}{0em}

\begin{center}

      \vfill
\includegraphics[width=\linewidth]{logo_heartbyte_transparent_v_1_1 (1)}

    \vfill
\clearpage

\end{center}
	\clearpage
	
	% Change Log
	
{\Large Document Change Record}

\begin{table}[h]
	\noindent\makebox[\textwidth]{
		\begin{tabularx}{1.2\textwidth}{l|l|w|n|n}
			\toprule
			\makecell[l]{Version \\ Number} & 
			\makecell[l]{Published \\ Date} & Description of Revision & Author & Approved by \\
			\midrule
			1.0 & 2020/10/15 &  Initial version &Emma Johansson & \\
			1.1 & 2020/11/12  & Add SQA activities, Document Review and Inspection process	& Emma Johansson & \\
			\bottomrule
		\end{tabularx}
	}
\end{table}


	\clearpage
	
	% Table of Contents
	\tableofcontents
	\clearpage
	
	\section{Introduction}
	User tests on the low fidelity prototype of HeartByte's system is to be performed to evaluate how well the prototype corresponds to the real needs and preferences of the end user. The information gathered to form an understanding and formulating the requirements that this prototype derives from have mainly come from the customer's, Region Östergötland's, contact persons regarding the project. Thus, the contact directly with the end user is usable to get input from those who are intended to use the system. The goal of the user tests on the prototype is to increase the project team's understanding of what the end user perceives as a functional and easily handled system. Furthermore, the tests aims to evaluate if the necessary and relevant information is shown in a way that is easy to survey.
	
	\section{Method}
	The focus of the test lies on usability, therefore the methods thinking out loud and System Usability Scale (SUS) are used. The user is presented with a number of tasks to do and is afterwards asked questions. The thinking out loud method is used by encouraging the user to express their thoughts and feelings during the test. The questions following the test is an abbreviated SUS-test, see \ref{Questions}. 
	
	The user test were performed with two nurses from Region Östergötland. The tests were held at a physical meeting where the users interacted with the system on a computer provided by HeartByte. The tests were recorded with both audio and screen capturing to enable afterwards analysis. 
	
	The test is divided into three different sections:
	
	\begin{enumerate}
		\item Inform the respondent thoroughly about the backstory to the project - why the system is created and its intended use. 
		\item Let the respondent perform the following tasks within the application in order to familiarise with the application and test its usability:
		\begin{enumerate}
			\item Find the latest notification from Gunilla Andersson and interact with her health data
			\item Find where to manually add a new weight measurement for Gunilla
			\item Find Gunilla's credentials (telephone number, address etcetera)
			\item Filter the patient list to only show diabetes patients.
			\item Filter the patient list to only show patient of high priority / urgent need of help.
			\item Find when Gunilla has her next appointment with a doctor.
			\item Find all the notifications for high priority patients.
		\end{enumerate}
		\item Let the respondent answer the questions stated in section \ref{Questions}.
	\end{enumerate}
	
	\subsection{Questions} \label{Questions}
	SUS gives a good idea of how usable the system is experienced by its intended users. The test is built up by a questionnaire containing ten questions regarding the system. Five of these questions are to be positive statements and the other five are to be negative. This kind of test is often good to use when questions regarding usability for a web application is to be answered. The perks of using the SUS-test standard is mostly its easy set up and good coverage of the usability aspect. The ten questions are to be answered on a scale from one to five, where one is to be interpreted as "I strongly disagree" and five as "I strongly agree". How the result from the test is calculated is to be found under section \ref{ref: Calculations}. 
	
	\begin{enumerate}
		\item I think that I would like to use this system frequently.
		\item I found the system unnecessarily complex.
		\item I thought the system was easy to use.
		\item I think that I would need the support of a technical person to be able to use this system.
		\item I found the various functions in this system were well integrated.
		\item I thought there was too much inconsistency in this system.
		\item I would imagine that most people would learn to use this system very quickly.
		\item I found the system very cumbersome to use.
		\item I felt very confident using the system.
		\item I needed to learn a lot of things before I could get going with this system.
	\end{enumerate}
	
	In this test of the lo-fi prototype the number of questions were reduced. This was done because it was considered hard to answer them when the test were not using a full web application. Four questions were removed, and the following were used: 
	
	\begin{enumerate}
		\item I think that I would like to use this system frequently.
		\item I found the system unnecessarily complex.
		\item I thought the system was easy to use.
		\item I think that I would need the support of a technical person to be able to use this system.
		\item I would imagine that most people would learn to use this system very quickly.
		\item I needed to learn a lot of things before I could get going with this system.
	\end{enumerate}

	The user tests were pursued in Swedish and the exact translation used is to be found in 
	
	\subsection{Calculation of the result}
	In order to get a result from the test, the answers from the respondents are validated. See the equations below for how this is done. The positive statements are numbered with uneven numbers $(1, 3, 5, 7, 9)$ and the negative statements are numbered with even numbers $(2, 4, 6, 8, 10)$. The resulting calculation is denoted $X_{i,Positive}$ for $ i = \{1,3,5,7,9\}$ and $X_{i,Negative}$ for $ i = \{2,4,6,8,10\}$. $Y_i = $ The answered number on question $ i = \{1,..,10\}$.
	
	\begin{equation}
		X_{i,Positive} = Y_i - 1 \;, \;for \;i = \{1,3,5,7,9\} \; \;else\;  X_{i,Positive} = 0
	\end{equation}
	
	\begin{equation}
		X_{i,Negative} = 5 - Y_i \;,\; i = \{2,4,6,8,10\}  \;else\;  X_{i,Negative} = 0
	\end{equation}
	
	\begin{center}
		The resulting number for each respondent is then calculated as: 
	\end{center}     
	
	\begin{equation}
		{Final\;result}_{\;one\;respondent} =  (\sum_{i=1}^{10} X_{i,Negative} + X_{i,Positive} ) \times 2,5 
	\end{equation}
	
	The result will end up between zero and 100. A result above 68 is seen as above average and above 86 is seen as good. These limits and numbers are statistically based on thousands of tests pursued in the past in various projects. To deliver a viable result, there are different opinions on how many respondents that is needed. It can be argued that a larger group of 50 respondents are needed or that two is enough since there is no correlation between the liability and number of respondents. In this test, two respondents will participate. 
	
	\section{Results}
	This section presents the results from the user test in terms of answers to the modified SUS-test and statements gathered using thinking out loud procedure.
	
	\subsection{SUS-test}
	The result on the SUS-test of both respondents are presented in table \ref{tab:SUS-result}. 
	\begin{table}[h]
		\begin{tabularx}{0.6\textwidth}{c|C C|c}
			\toprule
			Question & \multicolumn{2}{c|}{Respondent} & Average \\
			 & A & B &  \\
			 \midrule
			1 & 4 & 4 & 4 \\
			2 & 2 & 1 & 1.5 \\
			3 & 5 & 5 & 5 \\
			4 & 2 & 1 & 1.5 \\
			5 & 2 & 1 & 1.5 \\
			6 & 5 & 4 & 4.5 \\
			\bottomrule
		\end{tabularx}
	\caption{The result from the modified SUS-test}
	\label{tab:SUS-result}
	\end{table}
	
	\subsection{Thinking Out Loud}
	This section presents the suggestions for improvement that was expressed by both respondents.
	
	\begin{itemize}
		\item In the overview view under the tab home: By hovering over a notice, the user should be able to see which measured values. 
		
		\item In the patient overview under the tab "översikt", the measured value" activity level should be included, while "Alkohol" and "Rökning" should not.
		
		\item In the patient overview under the tab “översikt”: It must be clear during what period “physical activity” and “blood pressure” have been collected. 
		
		\item In the patient overview under the tab “översikt”, should include information about allergies or other care restrictions. 
		
		\item In the overview view under the home tab, the user should be able to sort based on disease type, latest date, name, updated by. It must also be clear which measured value the sorting is based on. 
		
		\item When the filter view is open, the user should be able to filter based on priority level. 
		
		\item In the patient view under the calendar tab, the calendar activities must be color-coded based on the activity type. 
		
		\item In the patient view under the calendar tab, the Calendar must be able to be linked to "timebooks" / “tidböcker”. 
		
		\item In the Overview view under the tab notise-log: there should be three symbols: 1. "warning for exceeded limit value" 2. "message" and 3. "Lack of measured value of the patient (due to forgotten or technical problems)" 

		\item Diagnosis needs to be clearly visible. Could be added after SSN of the patient.
		\item When getting a notice of approving a new measured value, it would be good to get several ways of contacting the patient.
		
		\item The measured values that need some measure to be taken should have the name of the responsible person if someone have responded to the notice.
		
		\item The list of earlier diagnoses is not that prioritized to see easily, should be moved further down on the screen.
	\end{itemize}
	
	\section{Conclusion}
	
	The suggestions for improvement were took into account and resulted in the following changes of the prototype and corresponding requirements. Some improvements were small comments not included in the list above, but resulted in changes in the product.
	
	\begin{itemize}
		\item text text
	\end{itemize}
	
	Ändrade notissymbol i “hem” för att vara konsekventa genom hela prototypen
	tog bort rökning och alkohol som förslag på filtrering ty uppfattades ej som användbart
	tog bort rökning och alkohol i översiktsinformationen om patienten
	lade till att man ska kunna se vilken filtrering som är aktuell, då det annars kan vara oklart om jag ser alla patienter eller bara en filtrerad del. 
	ändring i notisloggen: tog bort ikoner på läkare då det gav ett missvisande intryck att läkare var orsaken till just den typen av notis. Tog även bort chat-ikonen då chatfunktionen blev låg prio, och för tillfället inte längre är med i vår prototyp. 
	vid hantering av ett oväntade mätvärde lade vi till som förslag i pop-upen att man ska kunna tas till patientens kontaktuppgifter för att kunna ringa patienten och snabbt kunna kolla upp larmet.
	då ett larm blivit behandlat av någon i vårdpersonalen ska namnet på denne synas om andra personalmedlemmar går in på samma larm. Dvs vid svävande över notisen har vi ändrat från “mätvärde behandlas” till “mätvärde behandlas av Per Persson”
	i översikten möblerade vi om så att nuvarande och diagnoser skulle komma fram mer då de är mer aktuella än patientens diagnoshistorik. Vi lade till att de mätvärden som visas är ett snitt för de senaste veckan, då det tidigare var osäkert om det var det aktuella värdet eller ett målvärde. Det förstorades även upp för att bli lättare att avläsa. 
	i översikten lade vi även till Uppmärksamhetssymbolen då den på ett bra sätt ska visa relevant information som saknades i vår prototyp. 
	
	UMS ger information gällande “överkänslighet, tillstånd och behandlingar som, om de inte är kända för hälso- och sjukvårdspersonalen, medför allvarligt hot mot patientens liv eller hälsa. En ny nationell symbol, samlar all information gällande Varning, Observanda och Smitta. Varje del av symbolen innehåller specifik information.”
	Bytte ut fliken “hälsodeklaration” till Symptomskattning” då det var oklart vilken sorts deklaration skulle finnas där och vad den skulle ha för syfte. symptomskattning däremot ska kunna kombineras med mätvärden för att kunna jämföra symptom såsom feber med oregelbundna mätvärden och se om det finns en koppling. 
	(Nej föresten, punkten gällande utbytet av fliken stämmer inte! Det ändrade vi innan testet när vi visade prototypen för kunden tisdagen innan!)
	Ändrade rubrik från “ansvarig vårdgivare” till ansvarig vårdpersonal” ty vårdgivare är inte en person. Lade även till att tydligt visa vilken av de ansvariga är primär. 
	Lade till “diagnos” i rubriken för den patientspecifika sidan för att du oavsett var du är på patientsidan ska kunna få en grov uppskattning av vilken typ av patient du har att göra med. Ska undvika onödiga klick till översikten fram och tillbaka. 
	Lade till kommentaren att man ska kunna ändra storlek på kalendern och tabellen för att välja vilken man vill förstora upp mer. 
	Lade i kalender till valet att kunna växla mellan en dagsvy (för att få en överblick av vad som händer idag)och en löpande vy(för att kunna se alla framtida händelser utan tomrummet mellan dem).
	
	The result of the SUS-test need to be compared to results of future user tests to show how the usability changes with the development of the system. The reduced number of questions makes it hard to compute the actual SUS value.
	
	\clearpage
	\begin{appendices}
		\section{Questions Translated to Swedish}
		Below is the list of the statements being used in the test translated to Swedish. 
		\begin{enumerate}
			\item Jag tror att jag skulle vilja använda detta system ofta.
			\item Jag tycker systemet är onödigt komplext.
			\item Jag tycker systemet var lätt att använda.
			\item Jag tror mig behöva hjälp av en tekniskt kunnig och insatt person för att nyttja systemet. 
			\item Jag skulle tro att de flesta personer skulle lära sig att använda detta system mycket snabbt.
			\item Jag måste lära mig många olika saker innan jag kan använda systemet.
		\end{enumerate}
	\end{appendices}
	
	
\end{document}