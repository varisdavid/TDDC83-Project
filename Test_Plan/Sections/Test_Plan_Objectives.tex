
This document describes the testing done on the patient handling software produced by HeartByte for our customer Region Östergötland. This document will map the testing procedure as a whole and provide information to how the tests are selected, created and executed. There will also be definitions of the environment setup and tools used to ensure proper testing. This testplan will follow the structure provided by guru99 [1]. This is a living document which means that it will be updated over time to meet the current requirements.

\subsection{Scope}
\subsubsection{In Scope}

\begin{itemize}
  \item Functional Requirements
\end{itemize}


\begin{itemize}
    \item Non-Functional Requirements 
\end{itemize}

  Detailed version of requirements are located to in the SRS. 

  
  


\subsubsection{Out of Scope}
Anything outside of our requirements and functionality will not be tested. OpenEHR will only have output tests since we can't manage the DB ourselves. As of 2020/11/11 a big part of unit testing will be scrapped since almost all of our backend functionality comes from OpenEHR, which can't be tested.
\subsection{Quality Objective}
To ensure that our application will meet the requirements, user tests are performed by Region Östergötland and testing is done by us to ensure functionality. This covers both the Functional and non functional requirements. 

To ensure that Bugs and issues are identified using a bug tracking system in Gitlab and is continuously checked for bugs. The bug tracking is further presented in 2.4.

The project is going to use Continous Integration in Gitlab to ensure that our code meets the functionality standard set by our automatic tests. By ensuring our CI Heartbyte is looking to implement Continuous Delivery by doing small updates to the software, instead of large feature updates. This means as the commit gets pushed, our building of the app will be done as well as testing on that build. If the tests pass the code will be added to our production client. 
\subsection{Test Logistics}
\subsubsection{Who will test?}
The tests will be made by the Test Leader, Testers and Analysts. Primarily the Test Leader and Tester will make the tests but once the Analysis is done with their tasks, they will join the tests. The tests are then executed.
\subsubsection{When does tests occur?}
The Integration tests will occur once the following input is done:
\begin{itemize}
    \item Software is test ready
    \item A test specification is created
    \item The test Environment is complete
\end{itemize}
Since testing is done in iterations, we will write tests for the parts of the software that is testable, then as software functionality grows, more tests will be added in later iterations. This is called regression testing.

As Integration tests are tested and passed they are added to the CI for regression testing, meaning that the tests will be ran every time code is pushed to our main branch in Gitlab.
\subsection{Roles}
\begin{flushleft}
   \textbf{Test Leader}
    
    
    Gustav Karlsson
  
   \textbf{Tester}
   
    Hjalmar Svensson
    
    Mattis Bark
   
   
   \textbf{Quality Coordinator}
   
    Emma Johansson
     

\end{flushleft}


\clearpage