\subsection{Overview}
In this project we are using TDD as our methodology for the project. This means that after development is done tests are ran to ensure functionality of the application. If tests are not cleared the application is then changed to fit tests and then refactored.  

\subsection{Test Levels}
We have a focus on Integration Tests and System tests. We use OpenEHR for our backend and we won't be testing its functionality with Unit-tests, this could be implemented later on, but is not seen as a requirement. SRS are used for mapping the Integration tests and every test has a corresponding Requirement which gets cleared if the test is executed. These will be shown in the XXX.


\subsection{Suspensions Criteria and Resumption Requirements}
In this state, we are not sure when the full functionality is set to be done. We will therefore not set a suspension criteria due to the lack of time.

\subsection{Test Completeness}
\begin{itemize}
    \item Run Rate - 100 \% mandatory unless stated.
    \item Pass Rate - X \%
\end{itemize}
At what criteria is a test considered complete. Needs to be discussed with QA. 
