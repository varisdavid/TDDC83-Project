

\section{General processes}
In this section the general processes for learning and sharing information that is used throughout the project is described.
    \subsection{Workshops}
  When development methods include new tools, management or a team leader, depending on the specificity of the task, select a person responsible for the area who is to gain a deeper understanding about the fore-mentioned tool. The person mentioned is then in charge of making sure that the rest of the team that is supposed to work with the tool gains the required knowledge about it. This is done via workshops online where the responsible person goes through the basics of the tool.  A workshop could also include different brainstorming activities, where the participants are to come up with ideas regarding risks, company name, requirements for the system or tools to use for a specific system.
  
% Presentations
    \subsection{Presentations}
The management produces presentations to give during CEO meetings to educate the team in what has been done during the week and what is to be done during the upcoming weeks. The management is also supposed to highlight what in the process that is the most critical to get done as soon as possible. The presentations might also be used to educate the team in processes and tools that makes the work more effective. Example of tools that this may include is GitLab, GitLab Issues, React.js, Flask etc. The presentations could also be about how to handle documents and where to place documents in order for them to be easily found by project group members once they are needed in order to obtain certain information. 

% Resource Sharing
    \subsection{Resource sharing}
 Through sharing resources in Teams and GitLab the team tries to share the knowledge and progression within their work. This assures that the whole team gets to take part of each other’s knowledge and where they currently are in terms of progression. It also prevents two people from doing the same thing on different ends. 
 
 The process has starting according to plan where for example the UX-designer and Usability designer have shared their finished lo-fi prototype with the entire team through our shared resources in General in Microsoft Teams. A usability test of the lo-fi prototype has also been done together with the testing group and this test has also been shared with the entire group in the same way, making sure that the entire project group stays updated on the progress and latest information that has been acquired from the customer. 
 
% Group meeting & discussions
    \subsection{Group meetings and discussions}
The team is divided into different teams, both smaller cross-functional teams and teams where the whole group is working together on a specific task in regards to usability tests and development. The cross-functional teams makes it easier to share the information between the R\&D department and the P\&S department and makes sure that the information is shared and spread across the whole company. They are also good in making sure that we have all aspects of the software in mind when developing components for it. Through our weekly group meetings that are supposed to take place at least once every week we assure that we have a good communication in the project group where we can have emphasis on the current important matters.

%communication directives
    \subsection{Communication directives}


The management should mainly be in contact with the different team leaders as well as the architect and the integrator. All teams are to mainly have contact with each other within the teams, but other temporary chats are set up when working in small cross-functional teams. Examples of this is when the UX team as well as the testing team assures that the usability of the prototype is good through user tests with the end user. 

The communication is mainly to be done through chats in Microsoft Teams, since messages in a channel either pops up for the entire project group when tagging the channel or not being that visible for the people concerned when not tagging the channel.

For heavier tasks, meetings via Microsoft Teams should be held in order to have a more efficient work flow. Conclusions drawn from the meetings should be noted down for future reference and to have an easier time inserting changes into living documents.

The team leaders are to be responsible for communicating with other team leaders concerning correlated matters or when they are developing a software function which requires input from several groups like testing and development. Larger problems that are to be shared through the entire project group is to be first shared with management and there after shared by management via Microsoft Teams and via the project group meetings where the CEO is present. Sharing the information via two channels assures that the information is actually obtained by all the project members. 
