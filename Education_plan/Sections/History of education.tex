
    \section{History of education}
    The history of education section is supposed to give external stakeholders as well as project group members an overview of what education has been done within the project this far. 

\subsection{Individual education}
All members of the project have been educating themselves in useful techniques whenever need arose for new knowledge in order to implement the project's requirements. Education has been done continuously throughout the project where each individual responsible for a specific task has acquired the knowledge they need in order to effectively implement the tools that fits the task best. The developers have educated themselves in code writing practices, frameworks and different libraries. The design team has educated themselves in relevant best practices for design and design principles as well as the tools used to create prototypes. The testing team has educated themselves in tools used for testing such as Selenium, TDD and JUnit as well as learning how to do insightful usability tests with end users. The analysts are to continually improve their knowledge regarding how to write a good SRS that covers everything that the customer wants from the product.

\subsection{Education from management}
\begin{itemize}
\item \textbf{[Throughout the project]} Process document that describes what processes we are to use in the different teams aswell as best practice for those processes in regards to our project. 
\item \textbf{[Throughout the project]} Project plan that describes what is to be done during individual sprints and a GANTT-chart describing this in greater detail. The project plan also covered potential risks to the project, our goals, how the project team is set up and what every position is supposed to do as well as deliverables. 
 \item \textbf{[2020-09-11]} Workshop on how Azure DevOps works and how it is integrated with GitLab.
    \item \textbf{[2020-09-17]} Presentation of basic commands such cloning repo, pushing, pulling, merging in GitLab for the whole company. 
        \item \textbf{[2020-09-25]} Workshop on risks. Six group members brainstormed about what potential risks there might be for the project.

    \item \textbf{[2020-09-26]} Creation of guide on how to version handle LaTeX documents in Git.
    \item \textbf{[2020-11-12]} Workshop on how to be more explicit with inspection of documents as well as the change log to be more on the clear about how documents are revised and different versions of them. 
    \item \textbf{[Throughout the project]} Management presents the current status of the project during group meetings in order for all project members to get a good overview of the project.
    \item \textbf{[2020-12-02]} Management held a workshop for the experience summary for the VSSE. 
\end{itemize}
\subsection{Developement}
Developers have continuously educated themselves within their team in order to be able to provide a satisfying product before the code stop on the 4th of December.
\begin{itemize}
  \item \textbf{[Throughout the project]} An architecture notebook covering the architectural design of our system. This covered architectural decisions in regards to front- and back end frameworks as well as programming languages and architectural views (e.g. sequence diagrams, use cases and layers). 
    \item \textbf{[2020-10-21 - 2020-11-01]} For the testing team our integrator Emil Strömberg acquired a course on Udemy that is good for learning how to integrate tests into our continuous integration process. The concerned members of the project group went through this material during the exam period in order to gain the required knowledge before the development process started again after the exams. 
    
    \item \textbf{[2020-09-26]} Team leader Filip Eriksson, decided on some basic learning activities and material in order to secure basic knowledge across the development team. The first step in this process is to go through basic tutorials through react.js own website. There is also material that can be used through Code Academy, W3schools and Udemy that is good for getting the basic knowledge needed for the project. 
    \item \textbf{[2020-11-01 - 2020-12-04]} Continuous education in Yarn by Gustaf Eriksson for setting up the environment for developers. 
    \item \textbf{[Throughout the project]} READme in GitLab for setting up Yarn environment
    \item \textbf{[Throughout the project]} Guides in \href{https://teams.microsoft.com/l/file/615C62C2-B9BB-4CBA-BEA8-7DE9C1A7FAC5?tenantId=913f18ec-7f26-4c5f-a816-784fe9a58edd&fileType=docx&objectUrl=https%3A%2F%2Fliuonline.sharepoint.com%2Fsites%2FTeam_TDDC88_2020_C3%2FDelade%20dokument%2FGeneral%2FDevelopment%2FFramework%20for%20programming%20TDDC88.docx&baseUrl=https%3A%2F%2Fliuonline.sharepoint.com%2Fsites%2FTeam_TDDC88_2020_C3&serviceName=teams&threadId=19:2e5b2f7b1d9c4b17a70ddd8351b41898@thread.tacv2&groupId=ddbf5407-89}{Teams}}     that covers frameworks for development and guides for the learning process.
    \end{itemize}
    
    \begin{itemize}

    \item \textbf{[2020-10-05]} Released recording showing the Hi-Fi prototype being tested. This assures that developers have an easy time to acquire knowledge for how the design should look on the web application. Having the right design in the finished product is crucial to able to deliver a viable product to the customer at the VSSE on the 10th of December.
    
   \item \textbf{[2020-09-15 - 2020-09-30]} Introduction to user experience design - basic course online by Georgia Institute of Technology studied by Anna Wadsten
	\item \textbf{[2020-09-18 - 2020-10-05]} Anna Wadsten and Jessica Södereng watched tutorials of Figma where the material covered the basic components of Figma as well as a couple of advanced tips and tricks. 
\item \textbf{[Throughout the project]} Anna Wadsten and Jessica Södereng went through tutorials by DesignCourse (via youtube) - which described basic functions that can be used in Figma.
    \item \textbf{[Throughout the project]} Filip Eriksson have held workshops continuously for new people entering the development team.
    \item \textbf{[2020-11-01 - 2020-12-04]} Lukas Everborn and Axel Malmström have held workshops on the development process for front-end and back-end development respectively for new developers. 
 
\end{itemize}


\subsection{Testers}
Testers have been educating themselves in how to perform suitable tests for prototypes as well as for the web application. Performing relevant testing is important, since it allows the project to be sure that the finished product meets the requirements in the SRS. Meeting these requirements also means that the customer gets the product that they assume to be getting.
\begin{itemize}
  \item \textbf{[Throughout the project]} A test plan that covered the testing methodology, when tests will be done and who does the tests. The test plan also covered how the testing of deliverables is to be done and the environment and tools that are used for the testing. 
    \item \textbf{[2020-11-05 - 2020-11-12]} \href{(https://www.udemy.com/course/what-a-java-software-developer-must-know-about-testing/)} {Course on Udemy regarding testing, TDD, Junit and Selenium.} This course is meant to be complementary to the introductory lecture and labs we had on testing. The testing group felt that it was hard to grasp the essence of testing so we opted for a full course in this. After everyone took the course we had a meeting discussing it and went through if anyone had any questions. 
    \item \textbf{[2020-11-05 - 2020-11-12]} Watching a testing lecture on  \href{(https://www.youtube.com/watch?v=mRW2E8uweEc)}{youtube} regarding using Jest and selenium which is more project specific and went through some test cases in our own project. Tests were made together to make sure everybody in the group understood what was said in the lecture. 
    \item \textbf{[2020-11-05 - 2020-11-12]} An environment setup was made by Gustav Karlsson to make sure that all our testing is made from the same environment. As of now there is not a document on it but it will come. The testers shared their screens for installing, in order for Gustav to be able to troubleshoot any problems that could have arisen. 
    \item \textbf{[2020-11-16]} As a last step in the education, the testleader created a seminar on testing with Magnus Nielsen where he held a small presentation on what the essence of testing is and if we had any questions we could ask them in this seminar. Since our testers are already familiar with the framework and have started developing tests, more in depth questions could be asked. The day after the seminar the test group debriefed on what was said, and to see if there was any questions. 
        \item \textbf{[2020-11-17]} Gustav Karlsson have held a workshop in how to use the Kanban board in GitLab to easier track the workflow from requirement to setting up TDDs to developing to doing the final testing of the requirement. 
\end{itemize}

\subsection{Analysis}
\begin{itemize}
    \item \textbf{[Throughout the project]} Read a lot themselves about how SRS are supposed to be structured. Some relevant resources were shared with the group by Sara Lindholm. 
    \item \textbf{[Throughout the project]} Workshops on how to structure the SRS. 
       \item \textbf{[Throughout the project}] Sara Lindholm has been in contact with supervisor Sana and have communicated any conclusion drawn about the SRS with the rest of the analysts. This might have included introducing an overview in the form of a picture of the feature mentioned in the requirement. 
    \item \textbf{[Throughout the project}] Analysts have read a lot themselves about how the SRS is supposed to be structured.
    \item \textbf{[Throughout the project}] Continuous workshops by Sara Lindholm on how to structure the SRS. 

\end{itemize}

