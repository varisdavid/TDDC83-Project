\documentclass[12pt]{article}
\usepackage{authblk}
\usepackage{amsmath}
\usepackage{array}
\usepackage{hyperref}
\usepackage{ragged2e}
\usepackage{natbib}
\setlength{\parskip}{0em}

\usepackage[utf8]{inputenc}
\usepackage{subcaption}
\usepackage{natbib}
\usepackage{graphicx}
\usepackage[a4paper,width=165mm,top=25mm,bottom=25mm]{geometry}
\graphicspath{ {./Pictures/} }
\hypersetup{
    colorlinks=true,
    linkcolor=black,
    filecolor=magenta,      
    urlcolor=cyan,
}

\title{\Huge Education plan}

\author{Friberg M}


\begin{document}

% Title page
    
    \maketitle
\setlength{\parskip}{0em}

\begin{center}

      \vfill
\includegraphics[width=\linewidth]{logo_heartbyte_transparent_v_1_1 (1)}

    \vfill
\clearpage

\end{center}
%%
%%Daniels kommentarer: 
%Konkreta milstolpar och åtgärder för gruppens kompetensutveckling, gärna mappade mot projektplanens %milstolpar så att det blir tydligt hur kompetensen svarar mot behoven i projektplanen.
%Inte (bara) en önskelista över vad som behövs, utan vad som faktiskt ska göras: Vem ska göra vad %för att lära sig vad, på vilket sätt, när, varför, vem tar ansvar för det och hur följs det upp?

    

    

    \begin{center}
    
    
 \textbf{\large Change log}

\vspace{10mm}

\begin{tabular}{ | m{5em} | m{5em}| m{10em} |m{5em}| m{5em} |m{5em} |  } 
\hline
Version Number& Published Date & Description of revision & Author & Approved by \\ 
\hline
1.0 & 2020-09-22 & Sections about Workshops, Presentations, Resource sharing, Group meetings and discussions, Communication directives & Martin Friberg & Patrik Palmgren \\
\hline
1.1 & 2020-11-18 & Education to be done, education history and rewrote all the other sections. How to follow up on education & Martin Friberg & N/A \\
\hline
2.0 & 2020-11-23 & \textbf{Fixing inspection comments}. Adding section on individual education and fixing spelling errors as well as making general processes into subsections. Adding table of contents & Martin Friberg & Sam Anlér, Emma Johansson, Patrik Palmgren \\
\hline
2.1 & 2020-12-06 & Setting all the education to history. Adding the workshop for the experience summary. Adding education in terms of documents. & Martin Friberg & NA \\
\hline

\end{tabular}
\end{center}
            
\newpage
     
     % Table of contents, list of figures, and list of tables
    {
 
        \renewcommand{\contentsname}{Innehåll}
        \tableofcontents
    }
    
    
    \newpage
    

In this section, the purpose for the User Manual and an introduction to the website is detailed.

\subsection{Background}

This website was created to simplify the management of multiple patients at one operation.  

\subsection{Purpose}

The purpose of this manual is to in an easy way with help of screenshots from the website describe the different views and features of the website and how they work. 

\subsection{Introduction to website}

The website is divided into three different views, the login page, an overview page and one view for a single patient. The overview page contains information for a number of patients while the single patient page is for a specific patient. The overview page and the page for a single patient can not be reached unless you logged in before.
% Workshops


\section{General processes}
In this section the general processes for learning and sharing information that is used throughout the project is described.
    \subsection{Workshops}
  When development methods include new tools, management or a team leader, depending on the specificity of the task, select a person responsible for the area who is to gain a deeper understanding about the fore-mentioned tool. The person mentioned is then in charge of making sure that the rest of the team that is supposed to work with the tool gains the required knowledge about it. This is done via workshops online where the responsible person goes through the basics of the tool.  A workshop could also include different brainstorming activities, where the participants are to come up with ideas regarding risks, company name, requirements for the system or tools to use for a specific system.
  
% Presentations
    \subsection{Presentations}
The management produces presentations to give during CEO meetings to educate the team in what has been done during the week and what is to be done during the upcoming weeks. The management is also supposed to highlight what in the process that is the most critical to get done as soon as possible. The presentations might also be used to educate the team in processes and tools that makes the work more effective. Example of tools that this may include is GitLab, GitLab Issues, React.js, Flask etc. The presentations could also be about how to handle documents and where to place documents in order for them to be easily found by project group members once they are needed in order to obtain certain information. 

% Resource Sharing
    \subsection{Resource sharing}
 Through sharing resources in Teams and GitLab the team tries to share the knowledge and progression within their work. This assures that the whole team gets to take part of each other’s knowledge and where they currently are in terms of progression. It also prevents two people from doing the same thing on different ends. 
 
 The process has starting according to plan where for example the UX-designer and Usability designer have shared their finished lo-fi prototype with the entire team through our shared resources in General in Microsoft Teams. A usability test of the lo-fi prototype has also been done together with the testing group and this test has also been shared with the entire group in the same way, making sure that the entire project group stays updated on the progress and latest information that has been acquired from the customer. 
 
% Group meeting & discussions
    \subsection{Group meetings and discussions}
The team is divided into different teams, both smaller cross-functional teams and teams where the whole group is working together on a specific task in regards to usability tests and development. The cross-functional teams makes it easier to share the information between the R\&D department and the P\&S department and makes sure that the information is shared and spread across the whole company. They are also good in making sure that we have all aspects of the software in mind when developing components for it. Through our weekly group meetings that are supposed to take place at least once every week we assure that we have a good communication in the project group where we can have emphasis on the current important matters.

%communication directives
    \subsection{Communication directives}


The management should mainly be in contact with the different team leaders as well as the architect and the integrator. All teams are to mainly have contact with each other within the teams, but other temporary chats are set up when working in small cross-functional teams. Examples of this is when the UX team as well as the testing team assures that the usability of the prototype is good through user tests with the end user. 

The communication is mainly to be done through chats in Microsoft Teams, since messages in a channel either pops up for the entire project group when tagging the channel or not being that visible for the people concerned when not tagging the channel.

For heavier tasks, meetings via Microsoft Teams should be held in order to have a more efficient work flow. Conclusions drawn from the meetings should be noted down for future reference and to have an easier time inserting changes into living documents.

The team leaders are to be responsible for communicating with other team leaders concerning correlated matters or when they are developing a software function which requires input from several groups like testing and development. Larger problems that are to be shared through the entire project group is to be first shared with management and there after shared by management via Microsoft Teams and via the project group meetings where the CEO is present. Sharing the information via two channels assures that the information is actually obtained by all the project members. 


    \section{History of education}
    The history of education section is supposed to give external stakeholders as well as project group members an overview of what education has been done within the project this far. 

\subsection{Individual education}
All members of the project have been educating themselves in useful techniques whenever need arose for new knowledge in order to implement the project's requirements. Education has been done continuously throughout the project where each individual responsible for a specific task has acquired the knowledge they need in order to effectively implement the tools that fits the task best. The developers have educated themselves in code writing practices, frameworks and different libraries. The design team has educated themselves in relevant best practices for design and design principles as well as the tools used to create prototypes. The testing team has educated themselves in tools used for testing such as Selenium, TDD and JUnit as well as learning how to do insightful usability tests with end users. The analysts are to continually improve their knowledge regarding how to write a good SRS that covers everything that the customer wants from the product.

\subsection{Education from management}
\begin{itemize}
\item \textbf{[Throughout the project]} Process document that describes what processes we are to use in the different teams aswell as best practice for those processes in regards to our project. 
\item \textbf{[Throughout the project]} Project plan that describes what is to be done during individual sprints and a GANTT-chart describing this in greater detail. The project plan also covered potential risks to the project, our goals, how the project team is set up and what every position is supposed to do as well as deliverables. 
 \item \textbf{[2020-09-11]} Workshop on how Azure DevOps works and how it is integrated with GitLab.
    \item \textbf{[2020-09-17]} Presentation of basic commands such cloning repo, pushing, pulling, merging in GitLab for the whole company. 
        \item \textbf{[2020-09-25]} Workshop on risks. Six group members brainstormed about what potential risks there might be for the project.

    \item \textbf{[2020-09-26]} Creation of guide on how to version handle LaTeX documents in Git.
    \item \textbf{[2020-11-12]} Workshop on how to be more explicit with inspection of documents as well as the change log to be more on the clear about how documents are revised and different versions of them. 
    \item \textbf{[Throughout the project]} Management presents the current status of the project during group meetings in order for all project members to get a good overview of the project.
    \item \textbf{[2020-12-02]} Management held a workshop for the experience summary for the VSSE. 
\end{itemize}
\subsection{Developement}
Developers have continuously educated themselves within their team in order to be able to provide a satisfying product before the code stop on the 4th of December.
\begin{itemize}
  \item \textbf{[Throughout the project]} An architecture notebook covering the architectural design of our system. This covered architectural decisions in regards to front- and back end frameworks as well as programming languages and architectural views (e.g. sequence diagrams, use cases and layers). 
    \item \textbf{[2020-10-21 - 2020-11-01]} For the testing team our integrator Emil Strömberg acquired a course on Udemy that is good for learning how to integrate tests into our continuous integration process. The concerned members of the project group went through this material during the exam period in order to gain the required knowledge before the development process started again after the exams. 
    
    \item \textbf{[2020-09-26]} Team leader Filip Eriksson, decided on some basic learning activities and material in order to secure basic knowledge across the development team. The first step in this process is to go through basic tutorials through react.js own website. There is also material that can be used through Code Academy, W3schools and Udemy that is good for getting the basic knowledge needed for the project. 
    \item \textbf{[2020-11-01 - 2020-12-04]} Continuous education in Yarn by Gustaf Eriksson for setting up the environment for developers. 
    \item \textbf{[Throughout the project]} READme in GitLab for setting up Yarn environment
    \item \textbf{[Throughout the project]} Guides in \href{https://teams.microsoft.com/l/file/615C62C2-B9BB-4CBA-BEA8-7DE9C1A7FAC5?tenantId=913f18ec-7f26-4c5f-a816-784fe9a58edd&fileType=docx&objectUrl=https%3A%2F%2Fliuonline.sharepoint.com%2Fsites%2FTeam_TDDC88_2020_C3%2FDelade%20dokument%2FGeneral%2FDevelopment%2FFramework%20for%20programming%20TDDC88.docx&baseUrl=https%3A%2F%2Fliuonline.sharepoint.com%2Fsites%2FTeam_TDDC88_2020_C3&serviceName=teams&threadId=19:2e5b2f7b1d9c4b17a70ddd8351b41898@thread.tacv2&groupId=ddbf5407-89}{Teams}}     that covers frameworks for development and guides for the learning process.
    \end{itemize}
    
    \begin{itemize}

    \item \textbf{[2020-10-05]} Released recording showing the Hi-Fi prototype being tested. This assures that developers have an easy time to acquire knowledge for how the design should look on the web application. Having the right design in the finished product is crucial to able to deliver a viable product to the customer at the VSSE on the 10th of December.
    
   \item \textbf{[2020-09-15 - 2020-09-30]} Introduction to user experience design - basic course online by Georgia Institute of Technology studied by Anna Wadsten
	\item \textbf{[2020-09-18 - 2020-10-05]} Anna Wadsten and Jessica Södereng watched tutorials of Figma where the material covered the basic components of Figma as well as a couple of advanced tips and tricks. 
\item \textbf{[Throughout the project]} Anna Wadsten and Jessica Södereng went through tutorials by DesignCourse (via youtube) - which described basic functions that can be used in Figma.
    \item \textbf{[Throughout the project]} Filip Eriksson have held workshops continuously for new people entering the development team.
    \item \textbf{[2020-11-01 - 2020-12-04]} Lukas Everborn and Axel Malmström have held workshops on the development process for front-end and back-end development respectively for new developers. 
 
\end{itemize}


\subsection{Testers}
Testers have been educating themselves in how to perform suitable tests for prototypes as well as for the web application. Performing relevant testing is important, since it allows the project to be sure that the finished product meets the requirements in the SRS. Meeting these requirements also means that the customer gets the product that they assume to be getting.
\begin{itemize}
  \item \textbf{[Throughout the project]} A test plan that covered the testing methodology, when tests will be done and who does the tests. The test plan also covered how the testing of deliverables is to be done and the environment and tools that are used for the testing. 
    \item \textbf{[2020-11-05 - 2020-11-12]} \href{(https://www.udemy.com/course/what-a-java-software-developer-must-know-about-testing/)} {Course on Udemy regarding testing, TDD, Junit and Selenium.} This course is meant to be complementary to the introductory lecture and labs we had on testing. The testing group felt that it was hard to grasp the essence of testing so we opted for a full course in this. After everyone took the course we had a meeting discussing it and went through if anyone had any questions. 
    \item \textbf{[2020-11-05 - 2020-11-12]} Watching a testing lecture on  \href{(https://www.youtube.com/watch?v=mRW2E8uweEc)}{youtube} regarding using Jest and selenium which is more project specific and went through some test cases in our own project. Tests were made together to make sure everybody in the group understood what was said in the lecture. 
    \item \textbf{[2020-11-05 - 2020-11-12]} An environment setup was made by Gustav Karlsson to make sure that all our testing is made from the same environment. As of now there is not a document on it but it will come. The testers shared their screens for installing, in order for Gustav to be able to troubleshoot any problems that could have arisen. 
    \item \textbf{[2020-11-16]} As a last step in the education, the testleader created a seminar on testing with Magnus Nielsen where he held a small presentation on what the essence of testing is and if we had any questions we could ask them in this seminar. Since our testers are already familiar with the framework and have started developing tests, more in depth questions could be asked. The day after the seminar the test group debriefed on what was said, and to see if there was any questions. 
        \item \textbf{[2020-11-17]} Gustav Karlsson have held a workshop in how to use the Kanban board in GitLab to easier track the workflow from requirement to setting up TDDs to developing to doing the final testing of the requirement. 
\end{itemize}

\subsection{Analysis}
\begin{itemize}
    \item \textbf{[Throughout the project]} Read a lot themselves about how SRS are supposed to be structured. Some relevant resources were shared with the group by Sara Lindholm. 
    \item \textbf{[Throughout the project]} Workshops on how to structure the SRS. 
       \item \textbf{[Throughout the project}] Sara Lindholm has been in contact with supervisor Sana and have communicated any conclusion drawn about the SRS with the rest of the analysts. This might have included introducing an overview in the form of a picture of the feature mentioned in the requirement. 
    \item \textbf{[Throughout the project}] Analysts have read a lot themselves about how the SRS is supposed to be structured.
    \item \textbf{[Throughout the project}] Continuous workshops by Sara Lindholm on how to structure the SRS. 

\end{itemize}


\end{document}
