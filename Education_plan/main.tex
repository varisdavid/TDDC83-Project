\documentclass[12pt]{article}

\usepackage{authblk}


\usepackage[utf8]{inputenc}

\usepackage{subcaption}

\usepackage{natbib}
\usepackage{graphicx}
\usepackage[a4paper,width=165mm,top=25mm,bottom=25mm]{geometry}
\graphicspath{ {./Pictures/} }

\begin{document}

% Title page
    
    \maketitle
\setlength{\parskip}{0em}

\begin{center}

      \vfill
\includegraphics[width=\linewidth]{logo_heartbyte_transparent_v_1_1 (1)}

    \vfill
\clearpage

\end{center}
\clearpage

%Guidelines for learning
    \section{Guidelines for learning}
We have now decided on using React as our front-end development environment. Team leader Filip Eriksson, have decided on some basic learning activities and material in order to secure basic knowledge across the development team. The first step in this process is to go through basic tutorials through react.js own website. There is also material that can be used through Code Academy, W3schools and Udemy that is good for getting the basic knowledge needed for the project. 

For the testing team our integrator Emil Strömberg has acquired a course on Udemy that is good for learning how to integrate tests into our continuous integration process. The concerned members of the project group is to go through this material during the exam period in order to gain the required knowledge before the development process kick starts again after the exams. 

% Workshops
    \section{Workshops}
  When development methods include new tools, we select someone responsible for the area who is supposed to gain a deeper understanding about the fore-mentioned tool. The person mentioned is then in charge of making sure that the rest of the team that is supposed to work with the tool gains the required knowledge about it. This is done via workshops online where the responsible person goes through the basics of the tool.  A workshop could also include different brainstorming activities, where the participants are to come up with ideas regarding risks, company name, requirements for the system or tools to use for a specific system.
  
% Presentations
    \section{Presentations}
The management produces presentations to give during CEO meetings to educate the team in what has been done during the week and what is to be done during the upcoming weeks. The management is also supposed to highlight what in the process that is the most critical to get done as soon as possible. The presentations might also be used to educate the team in processes and tools that makes the work more effective. Example of tools that this may include is GitLab, GitLab Issues, React.js, Flask etc. The presentations could also be about how to handle documents and where to place documents in order for them to be easily found by project group members once they are needed in order to obtain certain information. 

% Resource Sharing
    \section{Resource sharing}
 Through sharing resources in Teams and GitLab the team tries to share the knowledge and progression within their work. This assures that the whole team gets to take part of each other’s knowledge and where they currently are in terms of progression. It also prevents two people doing the same thing on different ends. 
 
 The process has starting according to plan where for example the UX-designer and Usability designer have shared their finished lo-fi prototype with the entire team through our shared resources in General in Microsoft Teams. A usability test of the lo-fi prototype has also been done together with the testing group and this test has also been shared with the entire group in the same way, making sure that the entire project group stays updated on the progress and latest information that has been acquired from the customer. 
 
% Group meeting & discussions
    \section{Group meetings and discussions}
The team is divided into different teams, both smaller cross-functional teams and teams where the whole group is working together on a specific task in regards to usability tests and development. The cross-functional teams makes it easier to share the information between the R\&D department and the P\&S department and makes sure that the information is shared and spread across the whole company. They are also good in making sure that we have all aspects of the software in mind when developing components for it. Through our weekly group meetings that are supposed to take place at least once every week we assure that we have a good communication in the project group where we can have emphasis on the current important matters.

%communication directives
    \section{Communication directives}
Through set up communication directives every person in the group should know who their main contact is for sharing the information that they want to spread or to obtain the information that they want to acquire. The communication structure has also been set up in order to streamline the communication throughout the company. 

The management in mainly in contact with the different team leaders as well as the architect and the integrator. The team leaders are to be responsible for communicating with other team leaders concerning correlated matters or when they are developing a software function which requires input from several groups like testing and development. Larger problems that are to be shared through the entire project group is to be first shared with management and there after shared by management via Microsoft Teams and via the project group meetings where the CEO is present. Sharing the information via two channels assures that the information is actually obtained by all the project members. 
\end{document}